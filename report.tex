\documentclass{ds-report}

\assignment{Java RMI} % Set to `Java RMI`, `Java EE` or `Google App Engine`.
\authorOne{Konstantinos Stefanidis Vozikis} % Name of first team partner.
\studentnumberOne{r0123456} % Student number of first team partner.
\authorTwo{Sofie Landuyt} % Name of second team partner.
\studentnumberTwo{r1234567}  % Student number of second team partner.

\begin{document}
	\maketitle

	\paragraph{How would a client complete one full cycle of the booking process, for both a successful and
failed case? Base yourself on the example scenarios in Figure 1. Create sequence drawings to
illustrate this.\\} 
	Answer 1
	
	\paragraph{When do classes need to be serializable? You may illustrate this with an example class.\\} 
	A class needs to be serializable when an object of that class has to be transferred over a network. To illustrate this, we serialize the CarType class because it is the return type of a method of the ManagerSessionInterface.
	
	\paragraph{When do classes need to be remotely accessible (Remote)? You may illustrate this with  an example class.\\} 
	A class needs to be remotely accessible when certain methods of this class should be available to a client on another machine.
	Our AgencyInterface is remotely accessible, this way the Client class can invoke methods of the CarRentalAgency class that implements AgencyInterface.
	
	\paragraph{What data has to be transmitted between client and server and back when requesting the number
of reservations of a specific renter?\\}
	First, the client has to request new manager session.
	The data that is transmitted here is a ManagerSessionInterface from the Car Rental Agency server to the Client.
	Then, the client name is being transmitted from the client to the same server.
	After this, the Car Rental Agency queries the individual Car Companies. Now, the client name is being transmitted from the Car Rental Agency to every car company.
	The Car Companies each send their result, an integer, to the Car Rental Agency and then the Car Rental Agency sends the total result to the Client.
	
	
	\paragraph{What is the reasoning behind your distribution of remote objects over hosts? Show which
hosts execute which classes, if run in a real distributed deployment (not a lab deployment where everything runs on the same machine). Create a component/deployment diagram to illustrate this: highlight where the client and server are.\\}
	Answer 5
	
	\paragraph{How have you implemented the naming service, and what role does the built-in RMI registry play? Why did you take this approach?\\}
	Answer 6
	
	\paragraph{Which approach did you take to achieve life cycle management of sessions? Indicate why you picked this approach, in particular where you store the sessions.\\}
	
		
	\paragraph{Why is a Java RMI application not thread-safe by default? How does your application of synchronization achieve thread-safety?\\}
	Answer 8
	
	\paragraph{How does your solution to concurrency control affect the scalability of your design? Could synchronization become a bottleneck?\\}
	Answer 9
	
	\clearpage

	
	% You can include diagrams here.
	
\end{document}
